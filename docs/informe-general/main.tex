\documentclass[titlepage,a4paper,12pt]{article}
% Indica el estilo que se va a usar para todo el documento.
% Parámetros:
% a4paper, letterpaper, a5paper, …
% landscape: Apaisado
% titlepage: Hace que el tıtulo y el resumen queden en una página aparte. El resumen se indica con la instrucción \abstract{..}
% 10pt, 11pt, 12pt, … Tamaño de la letra.
% twoside, oneside. Simple o doble faz.
% twocolumn. Texto a dos columnas.
%
% Clases de documentos:
% article: Informes pequeños, trabajos prácticos.
% report: Informes largos, tesis, guiones. Tiene capítulos y apartados.
% book
% slide: Diapositivas

%%%%%%%%%%%%%%%%%%%%%%%%%%%%%%%%
% Paquetes
%%%%%%%%%%%%%%%%%%%%%%%%%%%%%%%%
\usepackage{color} % Paquete para darle color a la sintaxis del codigo fuente.

% Establece los márgenes de la hoja, aunque los margenes por defecto son bastante buenos.
\usepackage[top=2cm, bottom=2cm, left=2cm, right=2cm]{geometry} 

\usepackage{latexsym} % Este paquete permite usar simbolos especiales, no relacionados con la matemática, como  \Join o \Box
\usepackage{verbatim} % Para escribir codigo fuente.
\usepackage{amsmath} % La gran mayoría de los simbolos matemáticos
\usepackage{amssymb} % Algunos pocos símbolos matemáticos más raros, como \digamma
\usepackage{siunitx} % Notación exponencial
\usepackage{pdfpages} % Importar PDF

\usepackage[spanish]{babel} % Definimos el documento como que esta en español.

\usepackage[utf8]{inputenx} % Este paquete permite usar los acentos y eñes directamente en el texto.

\usepackage{graphicx} % Para usar imagenes
\usepackage{float}
\usepackage{courier}
\usepackage{listings} % Paquete para importar código fuente.

% Con esta instrucción definimos el interlineado. Por defecto es 1.
\linespread{1}

% Con esta instrucción obtenemos el número de página en el pie y una cabecera con el nombre de la sección (o con la sección en las páginas pares y la subsección en las impares si hemos indicado la opción twoside en el comando documenclass).
%\pagestyle{headings}

% Pero también está la instrucción \pagestyle{myheadings}, que pone el número de página al pie y en la cabecera pone el texto especificado por los comandos ``markboth{...}{...}'' y ``markright{...}''.
\pagestyle{myheadings}
%\markboth{Encabezado izquierdo}{Encabezado derecho} % Para doble faz
\markright{Trabajo Práctico Final: Dune 2000} % Para una carilla.

% Si no hemos especificado la opción twoside, todas las páginas se consideran derechas. Podemos cambiar el estilo de la página en curso mediante \thispagestyle. Por ejemplo, si queremos que la página en curso no tenga número escribimos \thispagestyle{empty} en el cuerpo del documeento.

%%%%%%%%%%%%%%%%%%%%%%%%%%%%%%%%
% Portada
%%%%%%%%%%%%%%%%%%%%%%%%%%%%%%%%

% En la instrucción \title{..}, se escribe el título del documento.
\title{ Trabajo Práctico Final: Dune 2000 \\
 \large{Informe General}}

\author{Iglesias, Matias \and Sportelli Castro, Luciano \and Alvarez Juliá, Santiago}

% Aquí podemos escribir la fecha de realización del trabajo práctico. La fecha actual se escribe con \today. Si no se quiere incluir la fecha, dejar la instrucción en blanco.
\date{ \today }

%%%%%%%%%%%%%%%%%%%%%%%%%%%%%%%%%%%%%%%%%%%
% AQUI COMENZAMOS EL DOCUMENTO
%%%%%%%%%%%%%%%%%%%%%%%%%%%%%%%%%%%%%%%%%%%

\begin{document}

%Lo primero que hacemos es crear el titulo.
\maketitle

%Creamos los indices que sean necesarios.
\tableofcontents %Índice general
%\listoffigures %Indice de imágenes
%\listoftables %Indice de tablas

\newpage
\section{División de tareas}

Basándonos en el enunciado del trabajo práctico nos dividimos las tareas de la siguiente manera:\\

% División de tareas (tarea, integrante)
\begin{center}
    \begin{tabular}{ | l | l |}
    \hline
    Tarea & Integrante \\ \hline
    Servidor & Matias Iglesias  \\ \hline
    Cliente & Luciano Sportelli Castro \\ \hline
    Editor & Santiago Alvarez Juliá \\
    \hline
    \end{tabular}
\end{center}

\section{Inconvenientes encontrados}

\subsection{Cliente}
El principal inconveniente encontrado en el cliente fue la dependencia de las texturas con el contexto de renderizado, que terminaron provocando dificultades a la hora de realizar un diseño adaptable. Este inconveniente está explicado con mayor detalle en el informe técnico.

\subsection{Servidor}
\subsubsection{Modelo}
Al intentar modelizar el juego y todos sus componenntes nos encontramos con algunos problemas que nos condicionaban mucho a la hora de querer implementarlo. Entre los mas destacados se pueden mencionar:
\begin{itemize}
\item El hecho de que teniamos que cargar la información referente tanto de las unidades como de los edificios desde un archivo externo.
\item La complejidad del juego y la cantidad de elementos que lo componen hace que sea muy sencillo caer en referencias cruzadas.
\item La elección de tratar de que cada objeto se tenga que instanciar lo menos posible(por ejemplo, en el caso de las armas) y que los mismos no se puedan acceder desde lugares incorrectos.
\item La correcta comunicacion enter el modelo-servidor-cliente. 
\end{itemize}

\subsection{Editor}

Un inconveniente encontrado a la hora de programar el Editor fue la manera en que tenía que ser almacenado un mapa en memoria para que el Editor mismo lo pueda cargar y también lo puedan leer el Cliente y el Servidor. La dificultad en este caso era conseguir almacenar el mapa de manera que ocupe la menor cantidad de espacio en memoria posible, lo que aumentaría la velocidad de lectura de éste, y que además sea fácil leerlo. La solución que encontramos fue almacenarlos con el formato JSON al ser un formato ampliamente conocido y sencillo de usar. Dentro del JSON se almacenan 2 arrays: uno representa la posición de cada jugador en el mapa y el otro representa al mapa en si mismo. El array de jugadores es de largo n siendo n la cantidad de jugadores. Cada elemento del array de jugadores a su vez es un array de largo 2 que representa la posición en el mapa del jugador, [X, Y]. El array que representa al mapa también es una array de arrays. En este caso los arrays que se encuentran dentro del array principal son de largo k siendo k la cantidad de columnas del mapa y el array principal tiene un largo de h siendo h la cantidad de filas del mapa. Dentro de los arrays secundarios se ubican strings que representas los diferentes terrenos ubicables en el mapa.

\section{Análisis de puntos pendientes}

\subsection{Cliente}
Entre los puntos pendientes del cliente quedaron: 
\begin{itemize}
\item Reestructurar el sistema de renderizado de modo de poder tener un almacenamiento en caché mucho más fuerte. 
\item Mejorar el sistema que evita que se reproduzcan sonidos simultáneamente de modo de hacerlo más inteligente.
\item Reestructurar el Área de Juego.
\item Ofrecer una pantalla de carga más informativa, que indique el estado de los otros jugadores.
\item Mejorar el renderizado del terreno, reutilizando los fragmentos de texturas anteriores.
\item Proveer un menu en partida para configurar opciones como el volumen de sonidos o la resolución de pantalla.
\item Agregar un chat de sala al lanzador del juego.
\item Proveer notificaciones push al lanzador del juego.
\end{itemize}

\subsection{Editor} 

Para mejorar la experiencia del usuario se podría agregar un feature para cuando hay que seleccionar un terreno de la pestaña Terrenos para poder ubicarlo en el mapa que facilite la ubicación de un mismo terreno en una zona grande del mapa. La idea sería que al mantener presionado el click izquierdo del mouse y moverlo sobre el mapa se ubicaría el terreno seleccionado de la pestaña en esa zona del mapa.\\

Otro feature interesante que haría al Editor más inteligente, que a su vez lo implementan editores de mapas reconocidos como el del videojuego Age of Empire, sería que en vez de mostrar distintas variaciones de un mismo tipo de terreno en la pestaña Terrenos, mostrar un único sprite por tipo de terreno y que un algoritmo se encargue de decidir cual sprite de cada tipo es el adecuado de ubicar en esa zona del mapa específica.  

\section{Herramientas}

Para programar las distintas aplicaciones requeridas por el TP se utilizaron las siguientes herramientas: 

% itemize segun tipo de herramienta

\begin{itemize}

\item Sistema de compilación: CMake.

\item Control de versiones: GitHub.

\item Editor de interfaz gráfica: Qt Designer.

\item Herramientas para debug: memcheck, callgrind, gdb.

\item Librerías gráficas: Qt, SDL.

\item Librerías generales: JSON para C++ modernos.

\end{itemize}

\end{document}